\documentclass{report}
\usepackage{graphicx}


\begin{document}
\title{Test Document}
\maketitle

\section{Introduction}
L’informatique est une ressource qui devient de plus en plus indispensable dans 
presque dans tous les domaines de la vie. Grace à son rapide développement 
nous assistons aujourd’hui à une véritable révolution technologique.

Ayant pris conscience des nombreux gains qu’offre l’informatique, les entreprises se sont 
aperçues qu’il est improbable voir impossible d’avoir une entreprise non informatisée.Ainsi l'enjeu pour ces entreprises ne serait pas uniquement de mettre en place des systèmes informatiques mais aussi il serait important de pouvoir s'assurer du bon fonctionnement de ces derniers.De ce fait,le suivi et la surveillance des systèmes informatiques sont des éléments essentiels de la gestion de l'infrastructure informatique dans toute organisation. 

C’est dans ce contexte que Flesk, qui est une entreprise spécialisée dans la conception et le développement de logiciels et d'applications mobiles sur mesure pour les entreprises, m’a accepté dans leur local pour une période de stage dans le but de développer un outil de monitoring qui permettra aux administrateurs systèmes de surveiller en temps réel la performance, la disponibilité et la santé des différents composants de l'infrastructure.

Ce rapport de projet de fin d’études en vue de l’obtention du diplôme de Master Professionnel
en Ingénierie des Systèmes d’Informations s’organisera en 4 chapitres : 

Le premier chapitre nommé cadre général du projet va présenter l’organisme d’accueil et le 
projet. Ensuite nous ferons une étude et critique de l’existant pour proposer notre solution. 

Pour le second chapitre intitulé analyse et spécifications des besoins, nous allons énoncer les 
différents acteurs y compris certains diagrammes avec les besoins fonctionnels et non 
fonctionnels de notre système.

Concernant le troisième chapitre intitulé conception, il s’agit de faire la modélisation de notre 
système avec certains diagrammes.

Enfin le dernier chapitre sera l’implémentation, il aura pour but de présenter l’environnement 
de développement et d’exposer les différentes interfaces développées pour notre système.
Pour finir, nous terminerons ce présent rapport en récapitulant le travail effectué puis 
montrerons les différentes perspectives possibles pour le projet.
\section{cadre general}
Introduction:
Dans ce chapitre, nous donnerons une description du projet et décrirons l’organisme d’accueil.
Ensuite nous ferons l’étude et la critique de l’existant qui consiste à étudier le système existant.

1. Présentation du projet:

1.1. Cadre du projet:
Notre projet s’inscrit dans le cadre de stage de projet de fin d’étude à la Faculté des Sciences de Monastir (FSM). Il s’est déroulé dans le local de la société Flesk Digital Group et a pour sujet la conception et le developpemnt d'un outil de monotoring.

1.2. Présentation de l’organisme d’accueil:

FLESK Consulting est une entreprise basée à Monastir en Tunisie qui offre des services de conseil et d'expertise en gestion, en informatique et en ingénierie. L'entreprise se concentre sur la fourniture de solutions sur mesure pour les entreprises et les organisations, en mettant l'accent sur l'amélioration de la productivité, de la qualité et de la performance globale.

FLESK Consulting propose une gamme de services de conseil, notamment en stratégie d'entreprise, en gestion de projet, en gestion des ressources humaines, en optimisation des processus, en sécurité informatique et en développement de logiciels. Elle s'efforce de fournir des solutions innovantes, efficaces et rentables pour aider ses clients à atteindre leurs objectifs commerciaux.

L'entreprise est composée d'une équipe de professionnels expérimentés et qualifiés, qui ont une solide expérience dans le domaine du conseil et de l'expertise en gestion. Elle s'engage à offrir des services de haute qualité à ses clients et à établir des relations à long terme basées sur la confiance et le respect mutuel.
 \includegraphics[scale=1]{../../../Downloads/OIP.jpg} 

//adresse
 Nom : Flesk
 Mail : contact@fleskConsulting.com
 Adresse : Rue sidi sarraj el omran Monastir
 Site Web : https://www.flesk.tn/
 Téléphone : +216 70 035 774
 
1.3. Outil de Monitoring:
La mise en place des systèmes d'information au sein des entreprises necessite un survi et une surveillance continue du materiel connécté.Cela dit il deviendra difficile pour un administrateur système de garantir cela dans une large topologie.d'ou nait le besoin de creer un outil de monitoring qui va assurer le suivi et la surveillance en temps reel et en continue du materiel connecté.

2. Etude et critique de l’existant:
L'entreprise utilise actuellement une solution qui interroge l’intranet,il s'agit de ‘OcsInventory’.C'est un logiciel libre de gestion d'inventaire et le déploiement de logiciels pour les réseaux informatiques. OCS Inventory permet de collecter des informations détaillées sur les matériels et les logiciels installés sur les ordinateurs et les serveurs connectés à un réseau.

Le logiciel est capable de fournir des rapports précis sur les informations collectées, tels que les caractéristiques matérielles des ordinateurs, les versions des logiciels installés, les mises à jour installées, etc. Ceci est très utile pour les administrateurs de systèmes informatiques, car cela leur permet de surveiller et de gérer efficacement les équipements informatiques de leur entreprise.

Cependant une solution limitée pour l'entreprise car OCS ne fait potentiellement que faire l’inventaire.

- Nous n’avons pas une traçabilité sur les bandes passantes perdues.
- Nous n’avons pas une traçabilité sur les fonctionnements inhabituels de notre
dispositif matériel et virtuel.
-Configuration initiale : La configuration initiale de OCS Inventory peut être un peu compliquée, en particulier pour les utilisateurs non familiers avec le système. Il peut être nécessaire de suivre des instructions détaillées ou de faire appel à un spécialiste pour aider à configurer le système correctement.

Complexité des rapports : Bien que OCS Inventory fournisse des rapports détaillés sur les informations collectées, ceux-ci peuvent parfois être difficiles à comprendre, en particulier pour les utilisateurs qui ne sont pas familiers avec les termes techniques de l'informatique.

Sécurité : OCS Inventory collecte des informations sur les ordinateurs et les serveurs connectés au réseau, ce qui peut soulever des préoccupations en matière de sécurité. Il est important de mettre en place les mesures de sécurité appropriées pour protéger les informations collectées par le système.
3. Solution envisagée:
Afin de pallier 	aux differents problèmes elevés précédement,et de resoudre d'autres besoins de la societé,nous avons proposé la mise en place d'une plateforme qui permettra d'utiliser l'inventaire de Osc afin d'effectuer la gestion des materiels connectés dans l'intranet et aussi consistera à effectuer d'autres taches telles que:
-la suivi de tous les dispositifs connectés et qui font parti de l'intranet ou extranet
-suiveiller tout le materiel reseau et autre
-Pouvoir detecter les anomalies reseaux,materiel,debit...
-Enregistrer les pings et les differentes interrogations sur les differents materiels(point d'accès,routeurs,vm,machines physiques,antenne....)
-Creer un dashboard de monitoring permettant de surveiller le materiel et VM.
4. Méthodologie de développement:
Le cycle de vie d’un logiciel est l’ensemble des phases de développement d’un logiciel à suivre
à partir de sa création jusqu’à sa disparition. Il en existe différents modèles dont :
• Modèle en cascade (Waterfall) : Le modèle en cascade est un modèle de développement de logiciel linéaire et séquentiel. Il est appelé "cascade" car les différentes étapes du développement sont organisées en une séquence linéaire, chaque étape étant terminée avant que la suivante ne commence.
 
\includegraphics[scale=1]{../../../Downloads/InfleXsys_MethodeDevCascade-480x356.png} 

• Modèle en V :
Le modèle en V est un modèle de développement de logiciel qui est basé sur le modèle en cascade. Le nom "V" vient de la forme en V que dessine le diagramme de la méthode.
Le modèle en V divise les étapes du développement en deux grandes phases. D'un côté, on trouve les phases de spécification, de conception et de validation, et de l'autre côté, les phases de réalisation, de tests unitaires, d'intégration et de validation système.

\includegraphics[scale=1]{../../../Downloads/R.jpg}
 
• Modèle en Agile :
Elle est une approche itérative et incrémentale qui s'oppose à la 
méthodologie en cascade. “Elle se veut plus souple et adaptée, et place les besoins du 
client au centre des priorités du projet.
//image
\includegraphics[scale=0.5]{../../../Downloads/fonctionnement-de-la-méthode-agile.jpg} 

4.1. Modèle en V:
comme nous l'avons vu precedement le Modèle en V est un modèle de développement de logiciel qui se divise en deux phases principales : la phase de spécification et la phase de validation et vérification.Plusieurs méthodologies peuvent être utilisées pour implémenter le modèle en V, voici quelques exemples :

*V-Modell XT : Cette méthode a été développée en Allemagne et est utilisée principalement dans le secteur public. Elle s'appuie sur des processus de développement normalisés et des modèles de qualité, et met l'accent sur la documentation détaillée à chaque étape.

*Rational Unified Process (RUP) : Cette méthode de développement logiciel est basée sur un processus itératif et incrémental. Elle fournit une structure pour organiser les activités de développement, de la planification initiale à la mise en production.

*Test Driven Development (TDD) : Cette méthode met l'accent sur l'écriture de tests automatisés avant même de commencer à écrire du code. Les tests définissent les spécifications et les exigences et guident le processus de développement.

*Agile en V : Cette méthode combine les principes du modèle Agile et du modèle en V pour fournir une approche flexible et itérative du développement de logiciel tout en suivant la structure de V.

*Model-Based Testing (MBT) : Cette méthode se concentre sur l'utilisation de modèles pour générer des cas de tests. Les tests sont exécutés pour vérifier que le logiciel produit répond aux spécifications du modèle.
\section{conclusion}
Par ce premier chapitre, nous avons donné une idée générale du projet après avoir présenté 
l’organisme d’accueil et nous avons fait une petite étude de l’existant. Dans le prochain chapitre 
nous allons essayer de dégager les fonctionnalités de notre système, passer à l’analyse et la 
spécification des besoins de notre projet.
\end{document}